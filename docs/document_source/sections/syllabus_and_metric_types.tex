\chapter{Syllabus Types}\label{ch:syllabus_and_metric_types}

With the Core Capabilties in mind, we move on to intentional design of the syllabi. We remind our audience that the calculation of L2M metrics seeks to answer a question that may differ from traditional reinforcement learning evaluation criteria, and thus, we enforce a rigid structure to evaluation tasks.
    
    
\section{Syllabus Types by Core Capability} 


\textit{Continual Learning}

    1) This syllabus consists of a single task with parametric variations (P1, P2, P3, ...)\\
        Can involve interpolation of parameters, noisy parameters, etc. but has only one task\\
    2) Each phase consists of a training block and an optional evaluation block\\
        Training block is training on one or more parametric variations \\
        Evaluation block allows Type II metrics to be computed\\
        
    Question: Can the agent adjust to changes in the environment?\\
        Metric: Recovery time (relative to previously trained parameter baseline)\\
            Calculated within and across training blocks \\
            After variation is introduced, does performance fall? Does it recover to the previous level? How quickly does it recover?\\
    
    Question: Can the agent maintain performance on previously learned parameters after being trained with new ones?\\
        Metric: Performance on previously learned parameters\\
            Recommended to be calculated within evaluation block\\

\textit{Adaptation to New Tasks}

    This syllabus consists of multiple, distinct tasks  (T1, T2, T3, ...). Parametric variations within a task are only exercised in Subtype C\\
    Each phase consists of a training block and an evaluation block\\
        Training block is training on one task at a time, and can be over some parametric variation, but only in Subtype C\\
        Evaluation block allows Type B metrics to be computed\\

    Subtype A: No expectation of skill transfer\\
        Parametric variations within a task are not exercised.\\
            Question: Can the agent learn a new task without forgetting the old task?\\
                Metric: Recovery time (relative to previous task baseline)\\
                    Calculated within and across training blocks \\
                    After a new task is introduced, does performance fall? Does it recover to the previous level? How quickly does it recover? \\
                Metric: Performance (reward statistics) on previously learned tasks\\
                    Calculated within evaluation blocks\\
                    Note that this is not to assess forward or reverse transfer; transfer is assessed in ANT Subtype B\\
                    
    Subtype B: Assess basic skill transfer\\
        Parametric variations within a task are not exercised.\\
        Question: Does the agent transfer knowledge from a learned task to a new task?\\
            Metric: Recovery time (relative to previous task baseline)\\
                Calculated within and across training blocks \\
                After a new task is introduced, does performance fall? Does it recover to the previous level? How quickly does it recover? \\
            Metric: Performance (reward statistics) on previously learned tasks\\
                Calculated within evaluation blocks\\
                Note that this is not to assess forward or reverse transfer; transfer is assessed in ANT Subtype B\\
        Metrics are:\\
            Time (number of episodes) to achieve threshold performance on Task 2 after being trained on Task 1\\
            Transfer matrix\\
            
    Subtype C: with Parametric skill transfer\\
        Syllabus has: Parametric variation in Task 1 and then see if that transfers to parametric variation in task 2.\\
        L2Arcade Example: vary paddle width in Pong; does this transfer to varying paddle width in Breakout? \\
            this can be formulated in both a forward and reverse transfer.\\
        Questions\\
            Transfer of general ability: does parametric variation in Pong improve Breakout time-to-learn at all\\
            Transfer of continual learning: Does parametric variation (continual learning) in Pong improve continual learning in Breakout?\\
        Metrics are:\\
            Amount of time required to learn vs from scratch \\
            
\iffalse           
    Compositional skill transfer
        we see this as an instance of 3c 
            Hava mentioned this as a prime goal for L2M program.
            Concern
                How do we identify the "skill relationships" between the tasks? Do we need to do this before designing the syllabus?
            Motivation: system learns Tasks T1, T2, T3 (each having one or more "skills"). Then in T4, system has to use a combination of skills from the prior tasks.
            What this might mean in case of L2Arcade
                T1 = Pong (skills: paddle control, intercept ball, steer ball away from opposing paddle)
                T2 = Breakout (skills: paddle control, intercept ball, steer ball toward blocks)
                T3 = Freeway (skills: control ball, avoid blocks)
                T4 = PongBreakout (skills: paddle control, intercept ball, steer ball toward blocks, and away from opposing paddle)
\fi

\section{Metric Definitions}

1. Saturation Value
    - Purpose: The saturation value is computed to quantify the maximum maintained value by the agent.\\
    - Calculated by: Since multiple rewards may be logged per episode, the mean is taken per episode. Then, the max of the rolling average is taken of the mean reward per episode with a smoothing parameter, s (default, 0.1)\\
    - Compared to: Future or single task expert saturation values of the same task\\
    - Computed for: CL, ANT\_A, ANT\_B\\                   

2. Time to Saturation
    - Purpose: Time to saturation is used to quantify how quickly, in number of episodes, the agent took to achieve the saturation value computed above.\\
    - Calculated by: The first time the saturation value (or above) is seen, that episode number is recorded\\
    - Compared to: Future times to saturation of the same task\\
    - Computed for: CL, ANT\_A, ANT\_B\\

3. Normalized Integral of Reward/Time
    - Purpose: Taking the Integral of Accumulated Reward over Time allows for a more robust comparison of the time to learn a particular task, taking into account both the shape and saturation of the learning for future comparison. Has limitations; must be normalized by length; only training phases can be compared to each other \\
    - Calculated by: Integrating reward over time, then dividing by the number of episodes used to accumulate the reward\\
    - Compared to: Future training instances of this metric\\
    - Computed for: ANT\_B\\

4. Recovery Time
    - Purpose: Recovery time is calculated to determine how quickly (if at all) an agent can "bounce back" after a change is introduced to its environment\\
    - Calculated by: After some training phase achieves a saturation value, determine how many episodes, if any, it takes for the agent to regain the same performance on the same task\\
    - Compared to: An agent's recovery time is comparable across tasks\\
    - Computed for: CL\\

5. Performance Maintenance on Test Sets\\
    - Purpose: Performance maintenance on test sets is calculated to determine whether an agent catastrophically forgets a previously learned task\\
    - Calculated by: Comparing all computed metrics on the train set (saturation values, time to saturation, etc) on the test set and computing the difference in performance\\
    - Compared to: N/A\\
    - Computed for: CL, ANT\_A, ANT\_B\\

6. Performance relative to STE (training) - Saturation Value, Time, Integral\\
    - Purpose: STE Relative performance assesses whether a lifelong learner outperforms a traditional learner.\\
    - Calculated by: Normalizing metrics computed on the lifelong learner by the same metrics computed on the traditional learner\\
    - Compared to: N/A\\
    - Computed for: CL, ANT\_A, ANT\_B\\

7. Forward Transfer (cross tasks)
    - Purpose: Forward transfer assesses whether an agent transfers knowledge from a learned task to a new task\\ 
    - Calculated by:\\ 
    - Compared to: STE performance\\
    - Computed for: CL, ANT\_A, ANT\_B\\
    
8. Forward Transfer (cross tasks)
    - Purpose: Forward transfer assesses whether an agent transfers knowledge from a learned task to a new task\\
    - Calculated by:\\
    - Compared to: STE performance\\
    - Computed for: CL, ANT\_A, ANT\_B\\

9. Time to Learn (cross tasks)
    - Purpose: Time to learn (cross tasks) assesses whether parametric variation in learning task 1 improve time to learn in task 2\\
    - Calculated by: \\
    - Compared to:\\
    - Computed for: ANT-C\\
    
10. Time to Continually Learn (cross tasks)
    - Purpose: Time to continually learn (cross tasks) assesses whether learning task 1 \\
    - Calculated by: \\
    - Compared to:\\
    - Computed for: ANT-C\\


\section{Example Syllabi}

Put up some JSON files here
